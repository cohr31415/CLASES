\documentclass[10pt,a4 paper]{article}
\usepackage[latin1]{inputenc}
\usepackage[spanish]{babel}
\usepackage{amsmath}
\usepackage{amsfonts}
\usepackage{amssymb}
\usepackage{dsfont}
\usepackage{parskip}
\usepackage[left=2cm,right=2cm,top=2cm,bottom=2cm]{geometry}
\usepackage{pstricks-add}
\usepackage{graphicx}
\usepackage{fancyhdr}
\usepackage{cite}


\pagestyle{fancy}

\begin{document}

\lhead{TRINOMIO VERANO 2020}
\chead{}
\rhead{Jemisson Daniel Coronel}

\begin{center}
\textbf{\Large{Problemas combinatoria 3}}
\end{center}

\begin{center}
Jemisson Daniel Coronel
\end{center}

\begin{center}
30 de enero de 2020
\end{center}
\vspace{9mm}
\begin{enumerate}

\item \textbf{(Problemas propuestos)} Hallar cu\'antas triplas de n\'umeros enteros no negativos ($a, b, c$) cumplen que su suma es $10$.   

\item \textbf{(Problemas propuestos)} Determinar de cu\'antas formas se pueden escoger dos subconjuntos disjuntos del conjunto $\left \{ 1, 2, 3,..., n \right \}$. 

\item \textbf{(Problemas propuestos)} Un juego llamado $kusigame$ es jugados por Ana y Beto. El juego consiste en un tablero de $4$ por $6$ con dos fichas, una para cada jugador, (una en el extremo superior izquierdo para el primer jugador y la otra en el extremo inferior derecho para el segundo jugador). El juego se realiza por turnos, en cada turno el jugador puede mover su ficha a una de sus casillas vecinas y el que visita una casilla ya visitada antes pierde. Si Ana empiece, ?`qui\'en tiene estrategia ganador? (si ambos juegadores quieren ganar). 

\item \textbf{(Problemas propuestos)} Diez $1's$ y diez $0's$ son escritos (en cualquier orden) alrededor de un c\'irculo. Probar que se puede escoger diez n\'umeros consecutivos tal que su suma sea exactamente 5. 

\item \textbf{(Problemas propuestos)} Los n\'umeros del $1$ al $10$ son escritos en la pizarra. Una operaci\'on consiste en borrar dos de ellos, supongamos que son $a$ y $b$, y escribir el n\'umero $ab + a + b$. Luego de $99$ operaciones es f\'acil notar de que solo queda un n\'umero. Probar que ese n\'umero que queda siempre es el mismo y hallarlo.

\item \textbf{(Olimpiada de mayo)} En cada escal\'on de una escalera de 10 pelda\~nos hay una rana. Cada una de ellas puede, de un salto, colocarse en otro escal\'on, pero cuando lo hace, al mismo tiempo, otra rana saltar\'a la misma cantidad de escalones pero en sentido opuesto (adem\'as ninguna rana se sale de los $10$ escalones). ?`Conseguir\'an las ranas colocarse todas juntas en el mismo escal\'on?  
\item \textbf{(Cono sur)} En una isla viven $13$ camaleones de color azul, $15$ de color rojo y $17$ de color verde. Si dos camaleones de diferente se encuentran, entonces ambos cambian su color al tercer color. ?`Ser\'a posible que en alg\'un momento todo los camaleones sean del mismo color?

\item \textbf{(ONEM tercera fase)} Determine el mayor n\'umero de cuatro d\'igitos $abcd$ que es m\'ultiplo de 4 y satisface la condici\'on $a < b < c < d$. 

\item \textbf{(ONEM segunda fase)} Determine cu\'antos n\'umeros capic\'uas de cinco d\'igitos son m\'ultiplos de $11$. 

\item \textbf{(ONEM cuarta fase)} Cada v\'ertice de un cubo de se pinta de rojo o de azul. Luego, cada cara del cubo se pinta de rojo, azul o moradode acuerdo a las siguientes reglas: se pinta de rojo si tiene m\'as v\'ertices rojos que azules, se pinta de azul si tiene m\'as v\'ertices azules que rojos o se pinta de morado si tiene $2$ v\'ertices de cada color. \\
a) ?`Ser\'a posible que al final dicho cubo tenga $3$ caras rojas y $3$ caras azules? \\
a) ?`Ser\'a posible que al final dicho cubo tenga $5$ caras moradas y 1 cara roja? \\



\end{enumerate}















\end{document}
