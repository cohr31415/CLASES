\documentclass[10pt,a4 paper]{article}
\usepackage[latin1]{inputenc}
\usepackage[spanish]{babel}
\usepackage{amsmath}
\usepackage{amsfonts}
\usepackage{amssymb}
\usepackage{dsfont}
\usepackage{parskip}
\usepackage[left=2cm,right=2cm,top=2cm,bottom=2cm]{geometry}
\usepackage{pstricks-add}
\usepackage{graphicx}
\usepackage{fancyhdr}
\usepackage{cite}



\pagestyle{fancy}

\begin{document}

\lhead{TRINOMIO VERANO 2020}
\chead{}
\rhead{Jemisson Daniel Coronel}

\begin{center}
\textbf{\Large{Problemas teor\'ia de n\'umeros 1}}
\end{center}

\begin{center}
Jemisson Daniel Coronel
\end{center}

\begin{center}
30 de enero de 2020
\end{center}
\vspace{9mm}
\begin{enumerate}

\item \textbf{(ONEM tercera fase)} Determine el mayor n\'umero de cuatro d\'igitos $abcd$ que es m\'ultiplo de 12 y satisface la condici\'on $a < b < c < d$. 

\item \textbf{(Problemas propuestos)} Sea $p$ un n\'umero primo. Probar que
$\binom{2p}{p}$ $\equiv$ $2$ ($mod$ $p^{2}$)

\item \textbf{(ONEM tercera fase)} Patricia escribi\'o un n\'umero de cuatro d\'igitos y luego insert\'o un d\'igito $5$ en la parte central, con lo cual obtuvo un n\'umero de cinco d\'igiots. Si la hacer esto el n\'umero original aument\'o en $1d500$, determine el valor de $d$.

\item \textbf{(Problemas propuestos)} Hallar todos los pares ($a, b$) de enteros positivos tales que:
$$20m + 12n = 2012$$

\item \textbf{(Cono sur)} Un entero positivo es llamado $guayaquilean$ si la suma de sus d\'igitos de $n$ es igual a la suma de d\'igitos de $n^{2}$. Hallar todos los posibles valores que puede tomar la suma de d\'igitos de un n\'umero $guayaquileana$.

\item \textbf{(ONEM segunda fase)} Los n\'umeros enteros positivos $a$, $b$, $c$ satisfacen las siguientes desigualdades:
$$a < 2b$$
$$b < 2c$$
$$c < 18$$
Determine el mayor valor posible de $a$.

\item \textbf{(ONEM segunda fase)} Determine cu\'antos enteros positivos $a$ cumplen que $a$ $<$ $8576$ y adem\'as:
$$mcd(a, 8575) = mcd(a + 1, 8575) = 1$$

\item \textbf{(ONEM cuarta fase)} Un conjunto formado por n\'umeros enteros positivos es llamado $super-divisible$ si la suma de sus elementos es divisible por cada uno de los elementos del conjunto. Determine cu\'antos elementos como m\'inimo debe tener un conjunto $super-divisible$ que continene a los n\'umeros $3$, $14$ y $21$. \\
$Aclaraci\'on$: Tenga en cuenta que un conjunto no tiene elementos repetidos. 


\item \textbf{(ONEM tercera fase)} Para cada entero $n$, sea $a_{n}$ el menor entero positivo tal que la suma de los cuadrados de sus d\'igitos es $n$. Por ejemplo, $a_{1} = 1$, $a_{2} = 11$, $a_{3} = 111$, $a_{4} = 2$ y $a_{5} = 12$. Sea $k$ un entero positivo y $d$ un d\'igito tal que $a_{k} = 13d6$, determine el valor de $k$.   

\item \textbf{(ONEM tercera fase)} ?`Cu\'al es el menor entero positivo tal que la suma de los cuadrados de sus d\'igitos es $23$? 


\end{enumerate}















\end{document}
