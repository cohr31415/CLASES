\documentclass[10pt,a4 paper]{article}
\usepackage[latin1]{inputenc}
\usepackage[spanish]{babel}
\usepackage{amsmath}
\usepackage{amsfonts}
\usepackage{amssymb}
\usepackage{dsfont}
\usepackage{parskip}
\usepackage[left=2cm,right=2cm,top=2cm,bottom=2cm]{geometry}
\usepackage{pstricks-add}
\usepackage{graphicx}
\usepackage{fancyhdr}
\usepackage{cite}



\pagestyle{fancy}

\begin{document}

\lhead{TRINOMIO VERANO 2020}
\chead{}
\rhead{Jemisson Daniel Coronel}

\begin{center}
\textbf{\Large{Problemas teor\'ia de n\'umeros 3}}
\end{center}

\begin{center}
Jemisson Daniel Coronel
\end{center}

\begin{center}
30 de enero de 2020
\end{center}
\vspace{9mm}
\begin{enumerate}

\item \textbf{(OIM)} Determine todas las parejas ($a, b$) de enteros positivos tal que:
$$a b^{2} + b + 7 | a^{2} b + a + b$$

\item \textbf{(ONEM tercera fase)} Calcule A:
$$a = \frac{1}{1003 \cdot 2004} + \frac{1}{1004 \cdot 2003} + \frac{1}{1005 \cdot 2002} ... + \frac{1}{2004 \cdot 1003}$$

\item \textbf{(ONEM tercera fase)} Sea $A$ = ($1 + 2$)($3 + 4$)...($99 + 100$). Encuentre el menor valor impar $N$, con $N$ $>$ $1$, tal que el m\'aximo com\'un divisor de $N$ y $A$ es $1$.  

\item \textbf{(ONEM cuarta fase)} Sean $a, b, c$ tres enteros positivos distintos tales que: 
$$a + 8b + 25c = 2004$$
Adem\'as $b$ es m\'ultiplo de $a$, y $c$ es m\'ultiplo de $c$. Encuentra todos los posibles valores de $a, b, c$.

\item \textbf{(ONEM tercera fase)} Si $p$ y $q$ son enteros positivos, tales que $\frac{5}{8}$ $<$ $\frac{p}{q}$ $<$ $$\frac{7}{8}$$. Encuentre el menor valor de $p$, si se debe cumplir que $p + q = 2005$. 

\item \textbf{(AOPS)} Halle todos los n\'umeros primos $p$, tal que $2^{p} + 3^{p}$ es divisible por $11$. 

\item \textbf{(ONEM segunda fase)} Sean $x$ y $z$ n\'umeros reales tales que
$$x^{2} + 5xz + z^{2} = 7$$
$$x^{2}z + z^{2}x = 2$$
Si $x + z$ $\neq$ $2$, determina el valor de $(6xz)^{2}$

\item \textbf{(ONEM segunda fase)} Sean $a$ y $b$ dos n\'umeros  enteros positivos tales que $a$ $>$ $b$ y el m\'inimo com\'un m\'ultiplo de $a$ y $b$ es $200$   

\item \textbf{(ONEM segunda fase)} ?`Cu\'antos n\'umeros enteros $x$ cumplen que $\frac{x^{3} + 2x^{2} + 9}{x^{2} + 4x + 5} $ es un entero? 

\item \textbf{(AOPS)} Halle todo los enteros naturales $n$ tal que tienen un divisor $d$ que cumple lo siguiente:
$$dn + 1 \mid d^{2} + n^{2}$$


\end{enumerate}















\end{document}
