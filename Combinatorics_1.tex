\documentclass[10pt,a4 paper]{article}
\usepackage[latin1]{inputenc}
\usepackage[spanish]{babel}
\usepackage{amsmath}
\usepackage{amsfonts}
\usepackage{amssymb}
\usepackage{dsfont}
\usepackage{parskip}
\usepackage[left=2cm,right=2cm,top=2cm,bottom=2cm]{geometry}
\usepackage{pstricks-add}
\usepackage{graphicx}
\usepackage{fancyhdr}
\usepackage{cite}


\pagestyle{fancy}

\begin{document}

\lhead{TRINOMIO VERANO 2020}
\chead{}
\rhead{Jemisson Daniel Coronel}

\begin{center}
\textbf{\Large{Problemas combinatoria 1}}
\end{center}

\begin{center}
Jemisson Daniel Coronel
\end{center}

\begin{center}
30 de enero de 2020
\end{center}
\vspace{9mm}
\begin{enumerate}

\item \textbf{(Revista Matematica Timisoara)} Sea $n$ un entero positivo impar mayor a 1. Probar que la secuencia de $\binom{n}{1}$, $\binom{n}{2}$,..., $\binom{n}{\frac{n - 1}{2}}$ contiene una cantidad impar de n\'umeros impares.

\item \textbf{(AMC12)} ?`C\'uantos enteros entre $1$ y $1000$ (inclusive) son m\'ultiplos de 3 pero no de 5?

\item \textbf{(AIME)} Hallar la cantidad de cu\'adruplas ordenadas ($x_{1}$, $x_{2}$, $x_{3}$, $x_{4}$) de n\'umeros enteros positivos impares que cumplan:
$$x_{1}+x_{2}+x_{3}+x_{4} = 16$$

\item \textbf{(AHSME)} 9 sillas en una fila son ocupadas por seis estudiantes y los profesores Andre, Beto y Carlos. Estos tres profesores van a llegar tarde, y los alumnos deciden escoger sus asientos de tal forma que a los costados (inmediatos) de cada profesor deben haber exactamente dos alumnos. ?`De cuantas formas se pueden escoger los asientos de los profesores Andre, Beto y Carlos?

\item \textbf{(Problemas propuestos)} ?`Cu\'antos n\'umeros de 5 d\'igitos que no poseen dos d\'igitos consecutivos de la misma paridad existen?  

\item \textbf{(AHSME)} Juan tiene un conjunto de 96 distintos bloques. Cada bloque es de uno de los 2 materiales (plastico o madera), 3 tama\~nos (peque\~no, mediano o grande), 4 colores (azul, verde, rojo o amarillo), 4 formas (circular, hexagonal, cuadrada o  triangular). ?`Cu\'antos bloques del conjunto son diferentes de "$plastico$ $mediano$ $rojo$ $circular$", en exactamente dos caracter\'isticas? (El "$madera$ $mediano$ $rojo$ $cuadrado$", es un ejemplo).

\item \textbf{(Problemas propuestos)} Un juego consiste en lanzar un dado varias veces y se gana el premio cuando hayan salido todos los posibles resultados del dado (es decir $1, 2, 3, 4, 5, 6$). ?`De cu\'antas formas se puede ganar en el s\'eptimo lanzamiento?

\item \textbf{(AHSME)} Llamamos a un n\'umero de 7 d\'igitos $d_{1}d_{2}d_{3}-d_{4}d_{5}d_{6}d_{7}$ $memorable$ si la secuencia $d_{1}d_{2}d_{3}$ es exactamente igual a $d_{4}d_{5}d_{6}$ o a $d_{5}d_{6}d_{7}$ (incluso puede ser a ambos). Asumiendo que todos los $d's$ pueden valer entre $0$ y $9$ (inclusive). Hallar cu\'antos n\'umeros memorables hay.


\item \textbf{(Problemas propuestos)} Se tiene un tablero de $2$ por $n$. ?`De cu\'antas formas se puede cubrir el tablero con exactamente $n$ dominos (tal que no haya dos que se superpongan)? 


\item \textbf{(Problemas propuestos)} En el alfabeto de Trinomiolandia se pueden usar solo las letras: $T$, $R$, $I$, $N$, $O$, adem\'as una palabra es legal en este pa\'is si no tiene dos vocales consecutivas o dos consonantes consecutivas. ?`Cu\'antas palabras legales hay en Trinomiolandia?





\end{enumerate}















\end{document}
