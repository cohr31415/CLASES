\documentclass[10pt,a4 paper]{article}
\usepackage[latin1]{inputenc}
\usepackage[spanish]{babel}
\usepackage{amsmath}
\usepackage{amsfonts}
\usepackage{amssymb}
\usepackage{dsfont}
\usepackage{parskip}
\usepackage[left=2cm,right=2cm,top=2cm,bottom=2cm]{geometry}
\usepackage{pstricks-add}
\usepackage{graphicx}
\usepackage{fancyhdr}
\usepackage{cite}



\pagestyle{fancy}

\begin{document}

\lhead{TRINOMIO VERANO 2020}
\chead{}
\rhead{Jemisson Daniel Coronel}

\begin{center}
\textbf{\Large{Problemas combinatoria 2}}
\end{center}

\begin{center}
Jemisson Daniel Coronel
\end{center}

\begin{center}
30 de enero de 2020
\end{center}
\vspace{9mm}
\begin{enumerate}

\item \textbf{(Problemas propuestos)} ?`C\'uantos enteros positivos de $5$ cifras hay tales que el producto de sus d\'igitos es $0$?

\item \textbf{(Problemas propuestos)} Hay $17$ n\'umeros escritos en la pizarra. Probar que hay $5$ ellos cuya suma es m\'ultiplo de $5$. 

\item \textbf{(Problemas propuestos)} Hay $5$ n\'umeros escritos en la pizarra. Probar que se pueden escoger algunos de ellos (al menos $1$) tal que su suma sea m\'ultiplo de 5.

\item \textbf{(Counting Strategies)} Juan quiere escoger tres n\'umeros distintos del conjunto $\left \{1, 2, 3, ..., 22 \right \}$ de tal forma que su suma sea m\'ultiplo de $3$. ?`De cu\'antas formas puede hacer eso? 

\item \textbf{(Problemas propuestos)} ?`Cu\'antos subconjuntos de $5$ elementos de $X$ = $\left \{1, 2, 3, ..., 24 \right \}$ tienen suma de elementos mayor o igual a $63$?   

\item \textbf{(Counting Strategies)} Hallar todos lo n\'umeros enteros positivos de dos d\'igitos que son divisibles por cada uno de sus d\'igitos ($12$ es un ejemplo).   

\item \textbf{(AIME)} 9 palitos son enumerados con $1, 2, 3, ..., 9$. Se escogen $3$ palitos (sin ver), ?`cual es la probabilidad de que la suma de esos $3$ n\'umeros sea impar? 

\item \textbf{(Theorem)} Probar que la siguiente igualdad cumple:
$$\binom{n}{k + 1} = \binom{n - 1}{k + 1} + \binom{n - 1}{k}$$

\item \textbf{(AIME)} Diez puntos son marcados en un c\'irculo. ?`De cu\'antas maneras distintas de pueden escoger algunos de esos puntos tal que formen un pol\'igono convexo? 

\item \textbf{(Counting strategies)} Sean $n$,$m$,$k$ enteros no negativos tales que $m$ es menor o igual a $n$. Probar que:
$$\binom{n}{k}\binom{k}{m} = \binom{n}{m}\binom{n - m}{k - m}$$



\end{enumerate}















\end{document}
