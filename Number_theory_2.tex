\documentclass[10pt,a4 paper]{article}
\usepackage[latin1]{inputenc}
\usepackage[spanish]{babel}
\usepackage{amsmath}
\usepackage{amsfonts}
\usepackage{amssymb}
\usepackage{dsfont}
\usepackage{parskip}
\usepackage[left=2cm,right=2cm,top=2cm,bottom=2cm]{geometry}
\usepackage{pstricks-add}
\usepackage{graphicx}
\usepackage{fancyhdr}
\usepackage{cite}



\pagestyle{fancy}

\begin{document}

\lhead{TRINOMIO VERANO 2020}
\chead{}
\rhead{Jemisson Daniel Coronel}

\begin{center}
\textbf{\Large{Problemas teor\'ia de n\'umeros 2}}
\end{center}

\begin{center}
Jemisson Daniel Coronel
\end{center}

\begin{center}
30 de enero de 2020
\end{center}
\vspace{9mm}
\begin{enumerate}

\item \textbf{(ONEM tercera fase)} Determine el menor entero positivo $N$ que tiene la siguiente propiedad. Al multiplicar $N$ por $45$ obtenemos un n\'umero tal que cada uno de sus d\'igitos est\'a en el conjunto $\left \{ 5, 7 \right \}$.  

\item \textbf{(IMO)} Para todo entero positivo $n$, prueba que la fracci\'on 
$$\frac{21n + 4}{14n + 3}$$
es irreductible.

\item \textbf{(Problemas propuestos)} Hallar todos los pares ($a, b$) de enteros positivos tales que $a$ $|$ $b + 1$ y $b$ $|$ $a + 1$.  

\item \textbf{(Problemas propuestos)} Sean $m$ y $n$ enteros positivos tales que $mn + 1$ es m\'ultiplo de $24$. Pruebe que $m + n$ tambi\'en es m\'ultiplo de $24$. 

\item \textbf{(Problemas propuestos)} Prueba que $n^{5}m - m^{5}n$ es divisible por $30$ para todo $n, m$ enteros positivos. 

\item \textbf{(Problemas propuestos)} Diremos que un n\'umero de $6$ d\'igitos es $especial$ si es imposible representarlo como el producto de un n\'umero de $3$ d\'igitos y otro de $4$ d\'igitos. Determine cu\'al es la m\'axima cantidad posible de n\'umeros consecutivosque son especiales. 

\item \textbf{(AOPS)} Pruebe que ($5n$)! es divisible por $5^{n}2^{3n}3^{n}$ para todo natural $n$.  

\item \textbf{(IMO)} Sean $p$ y $q$ PESI tal que:
$$1 - \frac{1}{2} + \frac{1}{3} - \frac{1}{4}+... + \frac{1}{1319} = \frac{p}{q}$$
Probar que $1979$ $|$ $p$.

\item \textbf{(ONEM segunda fase)} La suma de dos divisores positivos del n\'umero $45^{5}$ es 400, calcule la diferencia de dos divisores.  

\item \textbf{(ONEM segunda fase)} Decimos que dos enteros positivos son $amigos$ si su diferencia es un divisor de su suma. Por ejemplo, los n\'umeros $3$ y $5$ son amigos porque $2$ es un divisor de $8$. \\
Se tiene cuatro enteros positivos tales que cualesquiera dos de ellos son $amigos$. ?`Cu\'al es el menor valor que puede tomar la suma de esos cuatro n\'umeros?


\end{enumerate}















\end{document}
