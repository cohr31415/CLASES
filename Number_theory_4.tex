\documentclass[10pt,a4 paper]{article}
\usepackage[latin1]{inputenc}
\usepackage[spanish]{babel}
\usepackage{amsmath}
\usepackage{amsfonts}
\usepackage{amssymb}
\usepackage{dsfont}
\usepackage{parskip}
\usepackage[left=2cm,right=2cm,top=2cm,bottom=2cm]{geometry}
\usepackage{pstricks-add}
\usepackage{graphicx}
\usepackage{fancyhdr}
\usepackage{cite}



\pagestyle{fancy}

\begin{document}

\lhead{TRINOMIO VERANO 2020}
\chead{}
\rhead{Jemisson Daniel Coronel}

\begin{center}
\textbf{\Large{Problemas teor\'ia de n\'umeros 4}}
\end{center}

\begin{center}
Jemisson Daniel Coronel
\end{center}

\begin{center}
30 de enero de 2020
\end{center}
\vspace{9mm}
\begin{enumerate}

\item \textbf{(AOPS)} Si $a$ no es m\'ultiplo de un primo $p$ dado, probar que existe un entero positivo $b$ tal que $p^{b} - 1$ es m\'ultiplo de $a$. 

\item \textbf{(AOPS)} Probar que si $n$ es un n\'umero natural tal que $3n + 1$ y $4n + 1$ son ambos cuadrados perfectos, entonces $n$ es m\'ultiplo de 56. 

\item \textbf{(ONEM cuarta fase)} Decimos que un n\'umero natural es $variado$ si todos sus d\'igitos son distintos entre s\'i. Por ejemplo, los n\'umeros $9345$ y $1670$ son variados, pero $2007$ no lo es. Encuentre un n\'umero $N$ de $9$ d\'igitos tal que $N$ y $2N$ sean variados.   

\item \textbf{(ONEM tercera fase)} ?`Cu\'antas parejas ($a, b$) de enteros positivos cumplen las siguientes tres condiciones a la vez: \\
$-$ $a$ $>$ $b$. \\
$-$ $a - b$ es m\'ultiplo de 3. \\
$-$ $a$ y $b$ son divisores de $6^{8}$.

\item \textbf{(ONEM tercera fase)} Para cada entero positivo $n$ $>$ $1$, sea $p$($n$) el mayor divisor primo de n, por ejemplo $p$($20$) $=$ $5$, pues $5$ es el mayor divisor primo de $20$. Halla la suma de todos los valores de $n$ tales que:
$$n = 20p(n) + 2008$$

\item \textbf{(ONEM segunda fase)} Encuentra el mayor n\'umero de cinco d\'igitos distintos $abcde$ tal que $ab$, $bc$, $cd$ y $de$ sean n\'umeros primos. De como respuesta $a + b + c + d + e$. 

\item \textbf{(AOPS)} Halle todos los enteros $n$ tales que $n$! + $223$ sea un cubo perfecto. 

\item \textbf{(Problemas propuestos)} Probar que hay infinitos n\'umeros primos.  

\item \textbf{(ONEM cuarta fase)} Sean $a$, $b$, $c$ y $d$ cuatro n\'umeros enteros cuya suma es cero. Definimos 
$$M = (bc - ad)(ac - bd)(ab - cd)$$
Demuestre que M es un cuadrado perfecto.

\item \textbf{(Problemas propuestos)} Sea $a$ un entero positivo. Probar que $a^{2} + a + 1$ nunca es un cuadrado perfecto. 


\end{enumerate}















\end{document}
